% Conclusion sur le Process Hitting

\begin{comment}
  \frametitle{The Process Hitting modeling}

\todo{À réorganiser}

\begin{itemize}
  \item \tval{Dynamic} modeling with an \tval{atomistic} point of view
  \begin{fleches}
    \item Independent actions
    \item Cooperation modeled with cooperative sorts
  \end{fleches}

  \smallskip
  \item Efficient \tval{static analysis}
  \begin{fleches}
    \item Reachability of a process can be computed in \tval{polynomial time}\\
          \quad in the number of sorts
  \end{fleches}

  \smallskip
  \item Useful for the study of \tval{large biological models}
  \begin{fleches}
    \item Up to hundreds of sorts
    \item With few expression levels (Boolean or multivalued with $n \leq 4$)
  \end{fleches}

  \smallskip
  \item (Future) extensions
  \begin{fleches}
    \item Actions with priorities
    \item Continuous time with clocks?
  \end{fleches}
\end{itemize}

\end{comment}



\begin{frame}[c]
  \frametitle{Implementation in \Pint{}}

\begin{center}
  \tval{Existing free OCaml library:} \Pint
\end{center}

\medskip
\f Compiler + tools for Process Hitting models

\f Documentation \& examples: \lien{http://processhitting.wordpress.com/}

\pause
\bigskip
\medskip
\tval{Computation time for various reachability analyses:}

\medskip
\small
\begin{tabular}{r||c|c|c|c||c|c|c|}
\hline
\tval{Model} & Sorts & Procs & Actions & States & Biocham$^1$ & libddd$^2$ & \Pint \\\hline
\tval{\ex{egfr20}} & 35 & 196 & 670 & $2^{64}$ & [3s -- $\infty$] & [1s -- 150s] & \tval{0.007s} \\\hline
\tval{\ex{tcrsig40}} & 54 & 156 & 301 & $2^{73}$ & [1s -- $\infty$] & [0.6s -- $\infty$] & \tval{0.004s} \\\hline
\tval{\ex{tcrsig94}} & 133 & 448 & 1124 & $2^{194}$ & $\infty$ & $\infty$ & \tval{0.030s} \\\hline
\tval{\ex{egfr104}} & 193 & 748 &  2356 & $2^{320}$ &  $\infty$ & $\infty$ & \tval{0.050s}\\\hline
\end{tabular}

\medskip
\quad$^1$ Inria Paris-Rocquencourt/Contraintes\\
\quad$^2$ LIP6/Move

\citemodels

\end{frame}
