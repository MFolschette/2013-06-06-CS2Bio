% Définition du Réseaux Discrets Asynchrones

\newcommand{\Fadn}{\mathbb{F}}

\begin{frame}
  \frametitle{The Asynchronous Discrete Networks (ADN)}
  \framesubtitle{\tcite{\citedejong}}

\begin{itemize}
  \item The set of components: $N = \{ a, b, c, \dots \}$
  \item The local states (of one component $a$): $x^a \in \Fadn^a = \segm{0}{l_a}$
  \item The global states (of all components): $\Fadn = \Fadn^a \times \Fadn^b \times \Fadn^c \times \dots$  % \underset{a \in N}{\times}
  \item The evolution function (of one component $a$): $f^a : \Fadn \rightarrow \Fadn^a$
\end{itemize}

\begin{center}
\begin{tabular}{ccc}
  \begin{tabular}[b]{c|c|c}
    \multicolumn{2}{c|}{$X$} & \multirow{2}{*}{$f^a(X)$} \\
  \cline{1-2}
    $x^a$ & $x^b$ & \\
  \hline
    $0$ & $0$ & $1$ \\
    $0$ & $1$ & $0$ \\
    $1$ & $0$ & $0$ \\
    $1$ & $1$ & $0$
  \end{tabular}
&
  \begin{tabular}[b]{c|c|c}
    \multicolumn{2}{c|}{$X$} & \multirow{2}{*}{$f^b(X)$} \\
  \cline{1-2}
    $x^a$ & $x^b$ & \\
  \hline
    $0$ & $0$ & $1$ \\
    $0$ & $1$ & $0$ \\
    $1$ & $0$ & $0$ \\
    $1$ & $1$ & $0$
  \end{tabular}
&
  \begin{tabular}[b]{c|c|c}
    \multicolumn{2}{c|}{$X$} & \multirow{2}{*}{$f^z(X)$} \\
  \cline{1-2}
    $x^a$ & $x^b$ & \\
  \hline
    $0$ & $0$ & $0$ \\
    $0$ & $1$ & $1$ \\
    $1$ & $0$ & $1$ \\
    $1$ & $1$ & $2$
  \end{tabular}
\end{tabular}

\bigskip

\begin{tikzpicture}[adn]
  \path[use as bounding box] (-0.7,-0.7) rectangle (2.5,2);
  \node[inner sep=0] (z) at (2,0.75) {z};
  \node[inner sep=0] (a) at (0,1.5) {a};
  \node[inner sep=0] (b) at (0,0) {b};
  \path
    node[elabel, above=-1em of a] {$\segm{0}{1}$}
    node[elabel, below=-1em of b] {$\segm{0}{1}$}
    node[elabel, below=-1em of z] {$\segm{0}{2}$};
  \path
    (a) edge[loop left] (a)
    (b) edge[loop left] (b)
    (a) edge[bend right] (b)
    (b) edge[bend right] (a)
    (a) edge (z)
    (b) edge (z);
\end{tikzpicture}
\end{center}

\todo{Retirer les arcs vers $a$ et $b$ ?}

\end{frame}



\begin{frame}
  \frametitle{The Asynchronous Discrete Networks (ADN)}
  \framesubtitle{\tcite{\todo{}}}

\todo{Example de GE sur l'exemple ?}

Studying an ADN:

$$|\Fadn| = \prod_{a \in N} l_a$$

\f Exponential!

For Boolean models: $|\Fadn| = 2^{|N|}$

For multivalued models: … even more! $|\Fadn| \geq 2^{|N|}$

Some results can be found without the state graph

\todo{Bernot et al sur attracteurs cycliques et états stables}

But when it comes to reachability, the SG is mandatory

\end{frame}
